\documentclass[11pt]{article}
\usepackage[german]{babel}
\usepackage[utf8]{inputenc}
\usepackage[T1]{fontenc}
\usepackage{textcomp}
\usepackage{amssymb}
\usepackage{amsmath}
\usepackage[amssymb]{SIunits}
\usepackage{array,longtable}
\usepackage{booktabs}
\usepackage[a4paper,top=2cm,bottom=2cm,left=2.5cm,right=2.5cm,marginparwidth=1.75cm]{geometry}
\usepackage{amsmath}
\usepackage{mathtools}
\usepackage{fancyhdr}
\usepackage{hyperref}
\hypersetup{
    colorlinks=true,
    linkcolor=blue,
    filecolor=magenta,      
    urlcolor=cyan,
    pdftitle={Overleaf Example},
    pdfpagemode=FullScreen,
    }
%\usepackage{csquotes}
\pagestyle{fancy}
\fancyhead{} % clear all header fields
\fancyhead[RO,LE]{\textbf{Mathe Lernen Lernen Handout: Rekursion}}
\fancyfoot{} % clear all footer fields
\fancyfoot[LE,RO]{\thepage}
\fancyfoot[LO,CE]{Thanh Viet Nguyen}
\fancyfoot[CO,RE]{}
\renewcommand{\familydefault}{lmss}
\fontfamily{lmss}\selectfont
%\def \wide {.50\textwidth}
%\def \thin {.08\textwidth}
%\def \cbox {$\bigcirc$} .
\newcommand{\st}[1]{
	#1 & \cbox{} & \cbox{} & \cbox{} & \cbox{} & \cbox{}
}
\renewcommand*{\arraystretch}{1.3}

\begin{document}
	\title{Mathe Lernen Lernen: Rekursion}
	\author{ Thanh Viet Nguyen}
	\maketitle
	\thispagestyle{fancy}
    \section{Einleitung}
    \subsection{Was ist Rekursion}
    Der Begriff Rekursion weißt auf eine Definitionsweise von Funktionen, welche
    eine Funktion ist nämlich genau dann rekursiv, wenn der Funktionswert $f(n+1)$
    einer Funktion f:N→N von bereits errechneten Werten $(f(n), f(n-1), ...)$ abhängt.
    Ein Beispiel hierfür wäre die Fibonacci-Folge:
\newline 
$
f(n) = f(n-1) + f(n-2)
\text{mit}
f(0) = 0 und f(1) = 1
$
	\section{Zusatzaufgaben}
     
    \section{Quellen}
    \begin{itemize}
        \item \href{https://www.uni-ulm.de/fileadmin/website_uni_ulm/iui.prog/vortraege/2006_2007_pi_1_vortrag/Rekursion.pdf}{Uni Ulm: Rekursion}
    \end{itemize}

\end{document}